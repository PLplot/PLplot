\chapter{System dependent usage and installation}
\label{ap:sys}

Here are listed notes for PLPLOT usage and installation on various systems.
The user must have access to the proper PLPLOT library(s), utilities ({\tt
plrender} and optionally {\tt pltek}), and runtime access to the required
font files ({\tt plstnd4.fnt, plxtnd4.fnt}).  It is best to supplement these
notes with a site-dependent document outlining the exact procedure for using
PLPLOT.

Eventually we may make available the PLPLOT library binaries for different
machines via anonymous ftp, but for now users typically must build the
package themselves unless the site adminstrator has done so already.
Typically this will entail unpacking the .zoo or .tar PLPLOT source archive
on your system, making appropriate changes to set up the makefile or script
files, and then letting the computer take over.  A complete build of PLPLOT
will typically take only a few minutes on a fast workstation.  Systems
that are currently on the ``officially'' supported list should pose few
difficulties (we hope).  If you find that your system and/or compiler
require changes to the makefile and/or header files, feel free to contact us
about including these in the next PLPLOT release.

\section{Unix-type systems}
\label{sec:unix}

%%%%%%%%%%%%%%%%%%%%%%%%%%%%%%%%%%%%%%%%%%%%%%%%%%%%%%%%%%%%%%%%%%%%%%%%%%%%%

\subsection{Linking}

Linking your program with the PLplot library on a Unix type
system can be done using something similar to (this is system specific):
\begin{verbatim}
cc  -o main main.c libplplotf.a -lm -lX11     or
f77 -o main main.f libplplotf.a -lm -lX11
\end{verbatim}
for the single precision library ({\tt -lplplotd} for double).  
You can also link using the {\tt -l} option to the linker,
i.e.:
\begin{verbatim}
cc  -o main main.c -lplplotf -lm -lX11     or
f77 -o main main.f -lplplotf -lm -lX11
\end{verbatim}
however in this case you must be very careful to get the order of libraries
on the command line right (libraries to be searched last should be placed at
the end of the command line).  In the latter case the library ({\tt
libplplotf.a}) must be in the search path used by the linker or specified by
the {\tt -L} option.

The utilities {\tt plrender} and {\tt pltek} should be in your search path.
Some good places to put these include {\tt /usr/local/bin/} or {\tt
/usr/local/plplot/}.

The PLplot font files must be where the library can find them.
The current directory is always searched for the fonts first, followed
by a system-dependent (and perhaps site-dependent) search path.
This can be modified from the makefile, but is structured so that it won't
often need to be modified.  Note that directory names must be defined with
the trailing slash, if non-null.  The default search order for Unix-type
systems is as follows:
\begin{verbatim}
        current directory
        $(HOME)/lib/
        $(PLFONTS)
        PLFONTDEV1      (/usr/local/lib/)
        PLFONTDEV2      (/usr/local/lib/plplot/
        PLFONTDEV3      (/usr/local/plplot/)
\end{verbatim}

This is will not be a concern to the user if the fonts are located correctly
during installation.

%%%%%%%%%%%%%%%%%%%%%%%%%%%%%%%%%%%%%%%%%%%%%%%%%%%%%%%%%%%%%%%%%%%%%%%%%%%%%

\subsection{Installation}

To achieve a degree of flexibility and system independence for the PLplot
makefile, it is written in the macro language m4.  System dependencies are
resolved conditionally via m4 macros.  Using this method, only one file
({\tt makefile.m4}) is required for all systems with a SystemV-compatible
{\tt make}.  This enables us to easily support many different systems.

Here's how to make everything conveniently on a unix-like system.  First,
if you run into problems, you should read {\tt sys/unix/makefile.m4}
carefully, as it has much additional information inside.  If your system is
not already supported, you must add system specific defines analogous to
the existing ones in {\tt makefile.m4}.  Then: 
\begin{verbatim}
% cd tmp
% cp ../sys/unix/makefile.m4 .
% m4 -D<sys> makefile.m4 >makefile 
\end{verbatim}
where $\langle$sys$\rangle$ is an abbreviation for your system name, e.g.
SUNOS, UNICOS, AIX, etc.  To get the double precision library for your
system, specify {\tt -DDOUBLE} on the {\tt m4} command line (note: Fortran
double precision may not be supported on your system using this method;
check the release notes to be sure).  At this point, you may wish to edit
{\tt makefile} to change whatever settings are required {\em for your site
only}.  For example, you may wish to edit the device list to include only
those devices which you actually have access to. 

Then, while in the {\tt tmp} directory:
%
\begin{verbatim}
% make links
\end{verbatim}
%
sets up soft links to all the files you will ever need.
%
\begin{verbatim}
% make [libs]
\end{verbatim}
%
will make the PLplot library(s) ({\tt libs} is the first target and
therefore optional), whereas
%
\begin{verbatim}
% make everything
\end{verbatim}
%
makes the main library, fonts, {\tt plrender}, and {\tt pltek}, for the
default precision (probably single on a workstation).  This is just an
abbreviation for typing {\tt \% make libs fonts plrender pltek}.
To make one of the example programs,
%
\begin{verbatim}
% make x01c
\end{verbatim}
%
for the first C demo, or
%
\begin{verbatim}
% make x01f
\end{verbatim}
%
for the first Fortran demo (other example programs are similar).  To make
all the C program or Fortran program demos, specify {\tt cdemos} or
{\tt fdemos} as the target.

Finally, you must move the PLplot library, fonts, and utilities to 
a more permanent location, for example:
\begin{verbatim}
% mv ../lib/*.lib /usr/local/plplot
% mv *.fnt *.idx *.lkp /usr/local/plplot
% mv plrender pltek /usr/local/plplot
\end{verbatim}
if you are installing into system directories, or
\begin{verbatim}
% mv ../lib/*.lib ~/lib
% mv *.fnt *.idx *.lkp ~/lib
% mv plrender pltek ~/bin
\end{verbatim}
if you are installing in your directory tree only (note under SUNOS you must
run {\tt ranlib} on libraries after moving them).  You should then test the
library again using one of the example programs to ensure that everything
works perfectly.  After you are finished, you may delete everything in the
tmp directory.

Note that starting with PLplot version 4.0, the {\tt *.idx} and the {\tt
*.lkp} files are no longer generated.  The info they contained has been
stuffed into the {\tt *.fnt} files.  This means when you move the font
files to their permanent locale, you only need to do
\begin{verbatim}
mv *.fnt wherever
\end{verbatim}

If you have any difficutly with installing the PLplot files into their
permanent places, your local unix system wizard may be able to help you.

\section{VMS}

\subsection{Linking}

On VMS, the PLPLOT library is split into several parts due to limitations
of the VMS linker when used with the VAXC compiler.  These are:
\begin{verbatim}
libplplotf1.obj
libplplotf2.obj
\end{verbatim}
for the single precision library.  You will need to link with these
as well as {\tt sys\$library:vaxcrtl/lib}.

You will also need to make a symbol definition to allow {\tt plrender} to
run as it is meant to, if not already set up (note: {\tt pltek} currently
doesn't work on VMS).  This is done in the following way:
\begin{verbatim}
$ plrender :== $public:plrender.exe
\end{verbatim}
if {\tt public} is the logical name for the directory containing {\tt
plrender}.  Then {\tt plrender} will run the same as on Unix systems.

The PLPLOT font files must be where the library can find them.
The font locating code looks in the following places for the fonts:
\begin{itemize}
\item	current directory
\item	lib:
\item	sys\$login:	(home directory)
\item	PLFONTDIR	(a makefile macro, set when PLPLOT is installed)
\end{itemize}
This is will not be a concern to the user if the fonts are located
in PLFONTDIR as set during installation (defaults to
{\tt sys\$sysroot:[sysfont.plplot]}). 

One difficulty the user may encounter when running PLPLOT under VMS arises
from ideosyncracies of the VAX C compiler.  Namely, binary files as created
by a VAX C program are always of record format STREAM\_LF, while the
customary binary file transfer protocols (ftp, kermit) insist on ``Fixed
length 512 byte records''.  Note that DECNET copy works on all file record
formats and does not suffer from this problem.  

Thus, any file created by PLPLOT under VMS must be transformed to the more
standard VMS binary file format (fixed length records) before doing anything
with it.  This includes printing or transfer to another system.  Contrawise,
when transferring a PLPLOT metafile to VMS, you must first convert it to
STREAM\_LF format before rendering it with {\tt plrender}.  There are
several small, public domain utilities available to do this conversion, for
example the {\tt bilf} utility that comes with the {\tt zoo} archive program
(by Rahul Dhesi).  A copy of {\tt bilf.c} is distributed with PLPLOT in the
sys/vms directory in case you do not have it already.

\subsection{Installation}

On VMS the build is a bit complicated although a makefile is provided
(using Todd Aven's MAKE/VMS, a PD make-like utility).  If you do not have
MAKE/VMS installed on your system, you can either: (a) get a copy, (b)
rewrite the given makefile to work with DEC's MMS, or (c) get a copy of the
object files from someone else.  For further information, see {\tt
[.sys.vms]makefile}. 

Note: the X window driver is NOT included.  I do not know what libraries to
link with or how common they are on vaxen in general (I've used at least
one vax which does not have any X libraries on it).  If you come up with a
way to deal with this (portably), feel free to send your changes. 

\section{Amiga}

\subsection{Linking}

With the Lattice C compiler, linking to PLplot is accomplished by something
like (this depends on the compiler options you have used when creating the
library):
\begin{verbatim}
lc -o main -Lm+plplot.lib main.c
\end{verbatim}

The PLplot font files must be where the library can find them.  The current
directory is always searched for the fonts first, followed by {\tt
fonts:plplot/}, and {\tt plfonts:}, in that order, with one additional place
that can be specified from the makefile.  If you put them in {\tt
fonts:plplot/} you should never get a requester asking you to mount {\tt
plfonts:}.

For a color requester PLplot tries in a couple of places.  The palette
program used should open in the top-most screen (note that {\tt
sys:prefs/palette} shipped with the 2.0 operating system does not do this).
Under 2.0, {\tt sys:tools/colors} is used (standard).  Under 1.3, it is
recommended that you make avalable the {\tt palette} program that comes with
the WorkBench 1.3 enhancer package.  PLplot looks for this program in {\tt
tools:} and if it's not there it looks in {\tt sys:tools}, so you'll want to
assign {\tt tools:} to the directory in which the palette program resides.

\subsection{Installation}

The build for the Amiga is very similar to the one for Unix.  There are two
extra utilities you need for this to proceed flawlessly:
\begin{itemize}
\item	{\tt make}\qquad (PD version, ported from mod.sources posting)
\item	{\tt m4}  \qquad (also PD)
\end{itemize}

{\tt m4} is used to convert the master makefile.m4 into a makefile suited
for your system.  This handles system dependencies, compiler dependencies,
and choice of math library.  From the {\tt plplot/tmp} directory, you can
create the makefile via:
\begin{verbatim}
% copy /sys/unix/makefile.m4 ""
% m4 -DAMIGA <makefile.m4 >makefile
\end{verbatim}

The default is to use IEEE single precision math, with the library {\tt
libs:mathieeedoubbas.library} located at run-time.  By contrast, if you want
to use the Motorola Fast Floating Point library, you would build the
makefile via:
\begin{verbatim}
% m4 -DAMIGA -DMATH=FFP <makefile.m4 >makefile
\end{verbatim}

Eventually there will be switches for compiler as well (right now it is set
up for SAS/C; if someone makes the changes for their favorite compiler,
please send me a copy).

Then, while in the tmp directory:
\begin{verbatim}
% make links
\end{verbatim}
copies all of the source files to the current directory.  It is much more
efficient to use links instead if you are using the 2.0 operating system.
There is a script provided ({\tt makelinks}) for SKsh users that sets
up links to all the necessary files.

All the other targets (except for those dealing with the Fortran interface)
are identical to that under a Unix-like system; please see Section
\ref{sec:unix} for more detail.  The Fortran interface is not supported on
the Amiga at present, as the vendors of Fortran compilers currently use a
custom object file format which prevents linkage of Fortran programs with C
programs.

\subsection{IFF and printer driver}

The IFF driver will prompt for resolution, page size, filename and
orientation unless {\tt plspage} and {\tt plsfile} or {\tt plsfnam}
are used.

The printer driver creates a black and white graph on your preferences
supported printer.  It uses the preferences selected density and page size
limits and ignores most of the other stuff.  If you have selected ``ignore''
in the page size limit options then a full page graph is produced.  The
other options operate as described in the Amiga documentation.  At present,
only ``ignore'' and ``bounded'' produce aspect ratio correct plots (usually
unimportant unless x and y must have the same scaling e.g. for pie charts or
polar plots).  You can get very high quality plots with this driver.  A full
page, high resolution plot requires a lot of memory, however (an 8"x10" 300
dpi plot requires $(8*300)*(10*300)/8 = 900000$ bytes).  You can use the page
limits to reduce the memory requirements and still get high quality output.

\subsection{HP Plotter (or {\tt PLT:} device)}

The {\tt PLT:} device is an HP plotter compatible device handler written by
Jim Miller and Rich Champeaux.  It is freely redistributable, but it is not
included with this package.  It gives high quality output like the
preferences printer driver but with full preferences support.  Also, it
usually requires less memory (for full page plots) than the preferences
driver, but is slower.  Highly recommended if you have a color printer or
are memory strapped.  The {\tt PLT:} device accepts virtually all of the
standard HP-GL commands (probably the HP7470 is the best PLplot device
driver to use).

\subsection{Amiga window driver}

Written by Tony Richardson.  This provides a normal window with standard
intuition gadgets.  You can resize the window even while the program is
plotting in the window (see the ``Redraw Enabled'' section below).  If you
are making several graphs on separate pages, use the close gadget to
advance from one ``page'' to the next. 

The PLplot menu selections are:

\begin{itemize}
      \item Save Configuration\\
            Saves current window configuration (window size, screen type,
            screen depth, colors, resolution, etc.). The configuration
            is saved in {\tt s:Plplot.def} (only 54 bytes).

      \item Reset\\
            Resets the window to the configuration in {\tt s:Plplot.def}
            (or to a default config if this file doesn't exist).

      \item Maintain Plot Aspect\\
            If this is checked the plot aspect ratio is maintained as the
            window is resized (handy for polar plots, etc.). Default
            is unchecked in which case x and y are stretched
            independently to fit in the window.

      \item Redraw Enabled\\
            If this is checked, then the graphics commands are buffered
            in {\tt t:plplot.dat}. This file is used to redraw the plot when
            required. It's status is checked only when a new graph is
            started. The buffer is also used to create the ``Full Page''
            prints under the ``Print'' selection.

      \item Select Screen Type\\
            A submenu allows the user to select either ``Workbench'' or
            ``Custom''.

      \item Print\\
            There are three submenu options here. The ``Bitmap Dump''
            does just that (with full preferences support). The
            output is pretty jagged, but you can play around with
            the preferences stuff (scaling, smoothing, etc.) to
            improve things. The other two submenus are
            ``Full Page (Landscape)'' and ``Full Page (Portrait)''. This
            uses the graphics buffer file (see ``Redraw Enabled'' above)
            to create graphics output similar to the preferences
            driver. However the aspect ratio can not be maintained.
            Same preferences options are used as in preferences driver.

      \item Save Bitmap as IFF File\\
            Self explanatory. You can use this to save your images and
            then touch them up (do area fills, etc.) with your favorite
            paint program.

      \item Screen Format (this menu only appears on the custom screen).\\
	    Here you may select ``Interlaced'', ``High Resolution'', 
	    ``Number of Colors'', or ``Set Color Palette''.
\end{itemize}

\section{OS/2}

At the time of this writing (Fall '91), the current release of OS/2 is version
1.3.  Consequently the current implementation of PLplot is 16-bit in nature,
and was developed using Microsoft 16 bit compilers.  Like all the rest of 
the industry, this author is anxiously awaiting the arrival of OS/2 2.0 with
its accompanying 32-bit programming model.  So, rest assured that there will
be a port of PLplot to OS/2 2.0 with all possible speed.  For the time being
PLplot does work with OS/2 1.3, and has been used with OS/2 1.2 as well.  
This author
has no experience with OS/2 versions prior to version 1.2, so no guarantees 
can be
made on that.

See the README file in {\tt sys$\backslash$os2} for additional 
information not covered herein.

\subsection{Compiler Support}

The OS/2 implementation of PLplot was developed using Microsoft C 6.0
and Microsoft Fortran 5.0 (these compilers were used for the DOS port
as well).  The majority of the code is of course independent of any
vendor specific compiler peculiarities, as evidenced by the wide range
of supported platforms.  However, in the case of OS/2 and DOS there is 
one notable exception.
Specifically, the way Microsoft chose to implement string argument 
passing in Fortran 5.0.  The details are rather involved, so will not
be given here.  Suffice it to say that it was necessary to write a 
special set of Fortran language interface stubs just for use with 
Microsoft Fortran 5.  If you wish to use some other compiler, you may
very well find that the Unix stubs work better.  It is genuinely hard
to imagine how anyone could write a compiler in such a way that would
make string handling any more difficult than Microsoft did in Fortran 5.0.

Further note that the supplied makefile is written with the syntax of
Microsoft NMAKE version 1.11.  Please do NOT attempt to use versions of
Microsoft NMAKE prior to 1.11, as both 1.0 and 1.1 have been known to
dramatically fail, and even to trash source files (through the misapplication,
of default rules).  It should not be
difficult to transform the provided makefile for use with other vendor's
make facilities.  If you do so, please send in the changes so they can
be provided to other users in future releases of this software.

\subsection{Installation}

Compiling the libraries is not too much work if you have the compilers
listed above.  Just get into the {\tt plplot$\backslash$tmp } directory, 
and issue the following commands:

\begin{verbatim}
copy ..\sys\os2\makefile .
nmake links
nmake
nmake fonts
nmake plrender
nmake cdemos fdemos
\end{verbatim}

The {\tt fonts } target bears special mention.  It compiles the programs which
create the fonts, runs them, and copies the resulting font files to 
{\tt $\backslash$lib }
on the current drive.  You should make sure you have such a directory before
you begin this process, or else change the makefile.

Once done building the software, you'll probably want to copy the files
{\tt plplot.lib } and {\tt plstub.lib } to someplace like {\tt $\backslash$lib } 
and  the {\tt plrender } utility to someplace in your path.

For programs to run, they need access to the fonts.  This is most easily
accomplished by setting the {\tt plfonts } environment variable.  If you
leave the fonts in the default place, the command would be:
\begin{verbatim}
set plfonts=d:\lib\
\end{verbatim}
where d is the drive where you'll be keeping the fonts.

\subsection{Linking}

Once the libraries are built, linking with them is not too difficult at all.

\begin{verbatim}
cl /AL main.c plplot.lib
fl /AL prog.for plplot.lib plstub.lib
\end{verbatim}

You will of course need to have the various compiler and linker recognized
environment variables correctly specified.  A sample file 
{\tt plplot$\backslash$sys$\backslash$os2$\backslash$plconfig.cmd } is 
provided to show how I do this.  Here 
is the important portion of the file:
\begin{verbatim} 

set	 path=d:\c6.0\binp;d:\c6.0\binb;c:\f5.0\bin;%syspath%
set	 path=%path%d:\util;c:\bndutil;d:\cmd

set	  lib=d:\c6.0\lib;c:\f5.0\lib
set   include=d:\c6.0\include;c:\f5.0\include
set helpfiles=d:\c6.0\help\*.hlp
set	   cl= /FPi87 /Lp
set	   fl= /FPi87 /link /NOE /NOD:llibfor7 /NOD:llibc7 llibc7p llibf7pc

set   plfonts=d:\lib\

\end{verbatim}
whwere {\tt syspath} is an environment variable containing all the directories
needed for the system to operate normally.
For descriptions of each of the various command options, see the compiler
reference materials.  Note that PLplot is compiled using the large memory 
model.  Note also that to use Microsoft Fortran with PLplot you 
will need to have the C compatible Fortran libraries built.

\subsection{Presentation Manager Support}

Providing PLplot graphics under OS/2 PM proved to be a demanding project.
The basic problem is that in order for an OS/2 program to interact with
PM, it must be event driven and must do all its work in response to messages
which are sent to it by the PM.  In other words, it is simply not possible
with the current version of OS/2 (1.3 at the time of this writing) to 
write a program which operates in the normal procedural paradigm, and which
makes calls to OS/2 services to draw things on the screen---that's just
not the way it works.  Rather a program which uses PM to draw windows and 
images must set up a message queue, and then go into an infinite loop
in which it obtains messages from the queue and processes them.  These
messages are sent to the program by PM, and tell it to do things like ``resize
thyself'',  ``redraw thyself'', ``redraw this subsection of window such
and such'', and so forth.  This style
of programming is basically the exact inverse of how normal scientific and
technical programs are constructed.  Furthermore, OS/2 PM applications 
cannot write to {\tt stdout}.  These restrictions conspire to make it
very difficult to generate visual graphics from a ``normal'' program when
run on OS/2.  (The same is true of MS Windows, but due to the notorious
instability of that environment, no effort has been made to solve the
problem in that context--so far).

In spite of the substantial difficulties involved, an output capability for
OS/2 PM has been developed which allows the user of PLplot to be totally
unconcerned with the inner workings of a native PM code.  That is, the 
user/programmer may write programs which call plplot functions, just as
s/he would on any other platform, and it is possible to view the output
in a window.  {\tt stdout} is also made available, as one would expect,
so that in fact no special programming whatsoever is required to make
a ``normal'' PLplot program work on OS/2.  

Due to the substantial effort which was required to provide a transparent
and high quality graphics output capability on OS/2, this PM specific
code (and documentation) is not being provided under the same distribution
terms as the rest of the PLplot package.  Rather a small fee of \$15 for
individual users is required.  Note, the free distribution of PLplot will 
run on OS/2---it just won't come with the PM driver.

To obtain the add-on package which will allow PLplot programs to generate
output for OS/2 Presentation Manager, individual users should mail \$15. to:
\begin{quote}
Physical Dynamics \\
P.O. Box 8556 \\
Austin, Texas~~~78713 \\
\end{quote}
Companies or institutions with multiple users may obtain site licenses for
the following rates:
\begin{quote}
First 100 users, \$15 each. \\
Next 100 users, \$10 each.  \\
Additional users, \$5 each. \\
\end{quote}
Texas residents add 8\% sales tax.  When you send in your money, be sure 
to specify your preferred delivery mechanism.  Options are: postal mailing
of the floppy of your choice, and internet e-mail of uuencoded ZOO, ZIP, 
etc.  Just
state your preference.  For those who request e-mail delivery, please include
a postal address in case there is any problem with the networks.
Companies requesting mulitple user site licenses should include a contact.

In brief, the capabilities of the OS/2 PM driver are:
\begin{itemize}

\item	Windowed output.  Like the X Windows driver.

\item	Image redraws when the window is exposed or resized.  This is the most
	important way in which the OS/2 PM driver is superior to the
        X Windows driver.

\item	Optional automatic frame advance.  The other major point of 
	functionality which the X driver doesn't have  (yet).
        
\item	Color selection compatible with the X driver.

\item	Minimizable/Maximizable, and preserves window contents  (just a
	special case of item 2 above).

\end{itemize}

If you choose not to obtain the OS/2 PM option  from Physical Dynamics, you
can certainly still use the free version of PLplot on OS/2.  The included 
metafile output option in PLplot allows you to generate a PLplot native
metafile from any OS/2 PLplot application.  You can then switch to a DOS box
and view the metafile using the {\tt plrender} utility.  You will need to
build a DOS version of {\tt plrender} once, and store it in a place accessible 
to your DOS path.  In fact, the PLplot metafile driver was originally invented 
specifically to allow viewing output in the DOS box from programs run in
OS/2 native mode, prior to the development of the OS/2 PM driver.

Be sure to check for README files in the 
{\tt plplot$\backslash$sys$\backslash$os2} directory for any additional
information not available at the time of preparation of this manual.

\section{MS-DOS}

The DOS implementation of PLplot is based on the OS/2 version.  The main 
difference being that the DOS version has a VGA driver.  The VGA driver
uses calls to the Microsoft graphics functions supplied with C 6.0 and
QC 2.0 and above.  If you wish to use a different compiler which does not
support the Microsft graphics functions, you will need to make a new driver,
presumably based on the code structure of the existing Microsoft specific
driver.
This is not difficult however, so should not be perceived as a significant
obstacle.  Send in the new driver and they'll be incorporated into future
versions for everyone's benefit.

In particular, the compilers used to develop the DOS version are the same
as for the OS/2 version: Microsft C 6.0 and Fortran 5.0.  Further, the
installation procedure is identical, except get the makefile from
{\tt plplot$\backslash$sys$\backslash$dos}.  Note that it is not possible
to make a bound version of PLplot or of {\tt plrender} since the DOS VGA
driver contains code which cannot be bound.  

See the README file in {\tt sys$\backslash$dos} for information not
contained herein. 

\section{DOS 386}

PLplot can also be built using the 
DOS 386 port of GNU CC by DJ Delorie.  This version of of GNU CC 
generates code which requires a 386, and it uses a DOS extender to 
provide virtual memory.  Using this compiler it is possible to write
programs which are much much larger than will fit into the computer's
memory.  Details for how to obtain this compiler and build PLplot
are provided in the file
{\tt plplot$\backslash$sys$\backslash$dos386$\backslash$readme.1st}.
Basically the procedure is just like that for DOS and OS/2.

One advantage of the DOS 386 version of PLplot over the straight DOS version,
is that the DOS 386 version has a much improved display driver.  Specifically,
the DOS 386 driver understands how to operate most SVGA video cards, and
will display in 256 colors on such cards.  Consequently this driver is
able to provide support for {\tt plrgb} function, whereas the straight DOS
driver currently does not.

See the {\tt readme.1st} file for more information.

\section{LINUX}

PLplot can be built on LINUX.  For basic info, see the section on Unix-type
systems earlier.  There are however some additional considerations which
bear mentioning.

First, LINUX does not currently support any video drivers.  Consequently the
only output options which are present in the makefile by default are those
for printers and the metafile.  This situation will hopefully be resolved
once the X-11 port to LINUX is completed.  If X is already available for
LINUX when you read this, you should be able to just add {\tt -DXWIN} to
the {\tt PLDEVICES} macro in the makefile.  In that event, be sure to fix
the link step to link against the X11 libraries as well.

Secondly, at the time of this writing (Spring '92), there are some problems
with some forms of binary file i/o on LINUX.  Specifically, using the
kernel 0.95c+ and the GCC 2.1 compiler, it is not possible to correctly
generate the fonts for PLplot.  This would appear to be totally crippling
since every PLplot program must be able to load the fonts.  The best
solution to this problem right now is to build PLplot on DOS, and copy the
{\tt *.fnt} files over to LINUX.  Hopefully this problem with the
interaction of GCC 2.1 and LINUX will be resolved soon, so this may not be
a problem when you read this.  You are welcome to attempt to build the
fonts and see what happens.  If you observe that {\tt plstnd4.fnt} has a
file size other than {\tt 6424} then you can be sure the fonts were not
built correctly.  In that case, build them on DOS and copy them over.

