\section{DOS 386}

PLplot can also be built using the 
DOS 386 port of GNU CC by DJ Delorie.  This version of of GNU CC 
generates code which requires a 386, and it uses a DOS extender to 
provide virtual memory.  Using this compiler it is possible to write
programs which are much much larger than will fit into the computer's
memory.  Details for how to obtain this compiler and build PLplot
are provided in the file
{\tt plplot$\backslash$sys$\backslash$dos386$\backslash$readme.1st}.
Basically the procedure is just like that for DOS and OS/2.

One advantage of the DOS 386 version of PLplot over the straight DOS version,
is that the DOS 386 version has a much improved display driver.  Specifically,
the DOS 386 driver understands how to operate most SVGA video cards, and
will display in 256 colors on such cards.  Consequently this driver is
able to provide support for {\tt plrgb} function, whereas the straight DOS
driver currently does not.

See the {\tt readme.1st} file for more information.
