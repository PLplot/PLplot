\section{Amiga}

\subsection{Linking}

With the Lattice C compiler, linking to PLplot is accomplished by something
like (this depends on the compiler options you have used when creating the
library):
\begin{verbatim}
lc -o main -Lm+plplot.lib main.c
\end{verbatim}

The PLplot font files must be where the library can find them.  The current
directory is always searched for the fonts first, followed by {\tt
fonts:plplot/}, and {\tt plfonts:}, in that order, with one additional place
that can be specified from the makefile.  If you put them in {\tt
fonts:plplot/} you should never get a requester asking you to mount {\tt
plfonts:}.

For a color requester PLplot tries in a couple of places.  The palette
program used should open in the top-most screen (note that {\tt
sys:prefs/palette} shipped with the 2.0 operating system does not do this).
Under 2.0, {\tt sys:tools/colors} is used (standard).  Under 1.3, it is
recommended that you make avalable the {\tt palette} program that comes with
the WorkBench 1.3 enhancer package.  PLplot looks for this program in {\tt
tools:} and if it's not there it looks in {\tt sys:tools}, so you'll want to
assign {\tt tools:} to the directory in which the palette program resides.

\subsection{Installation}

The build for the Amiga is very similar to the one for Unix.  There are two
extra utilities you need for this to proceed flawlessly:
\begin{itemize}
\item	{\tt make}\qquad (PD version, ported from mod.sources posting)
\item	{\tt m4}  \qquad (also PD)
\end{itemize}

{\tt m4} is used to convert the master makefile.m4 into a makefile suited
for your system.  This handles system dependencies, compiler dependencies,
and choice of math library.  From the {\tt plplot/tmp} directory, you can
create the makefile via:
\begin{verbatim}
% copy /sys/unix/makefile.m4 ""
% m4 -DAMIGA <makefile.m4 >makefile
\end{verbatim}

The default is to use IEEE single precision math, with the library {\tt
libs:mathieeedoubbas.library} located at run-time.  By contrast, if you want
to use the Motorola Fast Floating Point library, you would build the
makefile via:
\begin{verbatim}
% m4 -DAMIGA -DMATH=FFP <makefile.m4 >makefile
\end{verbatim}

Eventually there will be switches for compiler as well (right now it is set
up for SAS/C; if someone makes the changes for their favorite compiler,
please send me a copy).

Then, while in the tmp directory:
\begin{verbatim}
% make links
\end{verbatim}
copies all of the source files to the current directory.  It is much more
efficient to use links instead if you are using the 2.0 operating system.
There is a script provided ({\tt makelinks}) for SKsh users that sets
up links to all the necessary files.

All the other targets (except for those dealing with the Fortran interface)
are identical to that under a Unix-like system; please see Section
\ref{sec:unix} for more detail.  The Fortran interface is not supported on
the Amiga at present, as the vendors of Fortran compilers currently use a
custom object file format which prevents linkage of Fortran programs with C
programs.

\subsection{IFF and printer driver}

The IFF driver will prompt for resolution, page size, filename and
orientation unless {\tt plspage} and {\tt plsfile} or {\tt plsfnam}
are used.

The printer driver creates a black and white graph on your preferences
supported printer.  It uses the preferences selected density and page size
limits and ignores most of the other stuff.  If you have selected ``ignore''
in the page size limit options then a full page graph is produced.  The
other options operate as described in the Amiga documentation.  At present,
only ``ignore'' and ``bounded'' produce aspect ratio correct plots (usually
unimportant unless x and y must have the same scaling e.g. for pie charts or
polar plots).  You can get very high quality plots with this driver.  A full
page, high resolution plot requires a lot of memory, however (an 8"x10" 300
dpi plot requires $(8*300)*(10*300)/8 = 900000$ bytes).  You can use the page
limits to reduce the memory requirements and still get high quality output.

\subsection{HP Plotter (or {\tt PLT:} device)}

The {\tt PLT:} device is an HP plotter compatible device handler written by
Jim Miller and Rich Champeaux.  It is freely redistributable, but it is not
included with this package.  It gives high quality output like the
preferences printer driver but with full preferences support.  Also, it
usually requires less memory (for full page plots) than the preferences
driver, but is slower.  Highly recommended if you have a color printer or
are memory strapped.  The {\tt PLT:} device accepts virtually all of the
standard HP-GL commands (probably the HP7470 is the best PLplot device
driver to use).

\subsection{Amiga window driver}

Written by Tony Richardson.  This provides a normal window with standard
intuition gadgets.  You can resize the window even while the program is
plotting in the window (see the ``Redraw Enabled'' section below).  If you
are making several graphs on separate pages, use the close gadget to
advance from one ``page'' to the next. 

The PLplot menu selections are:

\begin{itemize}
      \item Save Configuration\\
            Saves current window configuration (window size, screen type,
            screen depth, colors, resolution, etc.). The configuration
            is saved in {\tt s:Plplot.def} (only 54 bytes).

      \item Reset\\
            Resets the window to the configuration in {\tt s:Plplot.def}
            (or to a default config if this file doesn't exist).

      \item Maintain Plot Aspect\\
            If this is checked the plot aspect ratio is maintained as the
            window is resized (handy for polar plots, etc.). Default
            is unchecked in which case x and y are stretched
            independently to fit in the window.

      \item Redraw Enabled\\
            If this is checked, then the graphics commands are buffered
            in {\tt t:plplot.dat}. This file is used to redraw the plot when
            required. It's status is checked only when a new graph is
            started. The buffer is also used to create the ``Full Page''
            prints under the ``Print'' selection.

      \item Select Screen Type\\
            A submenu allows the user to select either ``Workbench'' or
            ``Custom''.

      \item Print\\
            There are three submenu options here. The ``Bitmap Dump''
            does just that (with full preferences support). The
            output is pretty jagged, but you can play around with
            the preferences stuff (scaling, smoothing, etc.) to
            improve things. The other two submenus are
            ``Full Page (Landscape)'' and ``Full Page (Portrait)''. This
            uses the graphics buffer file (see ``Redraw Enabled'' above)
            to create graphics output similar to the preferences
            driver. However the aspect ratio can not be maintained.
            Same preferences options are used as in preferences driver.

      \item Save Bitmap as IFF File\\
            Self explanatory. You can use this to save your images and
            then touch them up (do area fills, etc.) with your favorite
            paint program.

      \item Screen Format (this menu only appears on the custom screen).\\
	    Here you may select ``Interlaced'', ``High Resolution'', 
	    ``Number of Colors'', or ``Set Color Palette''.
\end{itemize}
