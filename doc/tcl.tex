\c tcl.tex
\c Geoffrey Furnish
\c 11 October 1994
\c
\c Document the PLplot Tcl API, and related programming issues.
\c
\c $Id$
\c %%%%%%%%%%%%%%%%%%%%%%%%%%%%%%%%%%%%%%%%%%%%%%%%%%%%%%%%%%%%%%%%%%%%%%%%%%%%

\node Tcl, Extended WISH, API, Top
\chapter{Using PLplot from Tcl}

PLplot version 5.0 introduces a new way of accessing the PLplot
library--via the Tcl scripting language.  blah blah.

Each of the PLplot calls available to the C or Fortran programmer are
also available from Tcl, with the same name and generally the same
arguments.  Thus for instance, whereas in C you can write:
% Insert a few more calls to make this a more realistic example.
\begin{example}
    plenv( 0., 1., 0., 1., 0, 0 );
    pllab( "(x)", "(y)", "The title of the graph" );
\end{example}
you can now write in Tcl:
\begin{example}
    plenv 0 1 0 1 0 0
    pllab "(x)" "(y)" "The title of the graph"
\end{example}

In order to accomodate the ubiquitous requirement for matrix oriented
data in scientific applications, and in the PLplot API in particular,
PLPlot 5.0 includes a Tcl extension for manipulating matricies in Tcl.
This Tcl Matrix Extension provides a straightforward and direct means
of representing one and two dimensional matricies in Tcl.  Many of the
PLplot Tcl API functions accept Tcl matricies as arguments.  For
insance, in C you might write:
\begin{example}
    float x[100], y[100];

    /* code to initialize x and y */

    plline( 100, x, y );
\end{example}
In Tcl you can write:
\begin{example}
    matrix x f 100
    matrix y f 100

    # code to initialize x and y

    plline 100 x y
\end{example}

In general then, you can assume that each C function is provided in
Tcl with the same name and same arguments (and one or two dimensional
arrays in C are replaced by Tcl matricies).  There are only a few
exceptions to this rule, generally resulting from the complexity of
the argument types which are passed to some functions in the C API.
Those exceptional functions are described below, all others work in
the obvious way (analogous to the above example).  See the Tcl
example programs for extensive demonstrations of the usage of the
PLplot Tcl API.

\begin{menu}
* Tcl Matrix::		All about the Tcl Matrix extension
* Tcl plcont::		Drawing contour plots from Tcl.
* Tcl plshade::		Drawing color fill plots from Tcl.
\end{menu}

\c tclmatrix.tex
\c Geoffrey Furnish
\c 11 October 1994
\c
\c All about the Tcl Matrix extension facility.
\c
\c $Id$
\c %%%%%%%%%%%%%%%%%%%%%%%%%%%%%%%%%%%%%%%%%%%%%%%%%%%%%%%%%%%%%%%%%%%%%%%%%%%%

\node Tcl Matrix, Tcl plcont, Tcl, Tcl
\section{The Tcl Matrix Extension}

blah blah

\begin{menu}
* Tcl Usage::		Using Tcl matricies from Tcl.
* C Usage::		Using Tcl matricies from C.
* C++ Usage::		Using Tcl matricies from C++.
\end{menu}

\node Tcl Usage, C Usage, Tcl Matrix, Tcl Matrix
\subsection{Using Tcl matricies from Tcl}

blah blah

\node C Usage, C++ Usage, Tcl Usage, Tcl Matrix
\subsection{Using Tcl matricies from C}

blah blah blah

\node C++ Usage, , C Usage, Tcl Matrix
\subsection{Using Tcl matricies from C++}

is much easier than using them from C.  blah blah



\node Tcl plcont, Tcl plshade, Tcl Matrix, Tcl
\section{Drawing contour plots from Tcl}

isn't hard.

\node Tcl plshade, , Tcl plcont, Tcl
\section{Drawing color fill plots from Tcl}

isn't hard either.