\chapter {Introduction}
\pagenumbering{arabic}
\section {The PLPLOT Plotting Library}

PLPLOT is a library of C functions that are useful for making scientific
plots from a program written in C, C++, or Fortran.  The PLPLOT library can be
used to create standard x-y plots, semilog plots, log-log plots, contour
plots, 3D plots, mesh plots, bar charts and pie charts.  Multiple graphs (of
the same or different sizes) may be placed on a single page with multiple
lines in each graph.  Different line styles, widths and colors are
supported.  A virtually infinite number of distinct area fill patterns may
be used.  There are almost 1000 characters in the extended character set.
This includes four different fonts, the Greek alphabet and a host of
mathematical, musical, and other symbols.  The fonts can be scaled to any
desired size.  A variety of output devices are supported and new
devices can be easily added by writing a small number of device dependent
routines.

Many of the underlying concepts used in the PLPLOT subroutine package
are based on ideas used in Tim Pearson's PGPLOT package originally
written in VAX-specific Fortran-77.  Sze Tan of the University of
Auckland originally developed PLPLOT on an IBM PC, and subsequently
transferred it to a number of other machines.  Additional features were
added to allow three-dimensional plotting and better access to low-level
routines. 

The C version of PLPLOT was developed by Tony Richardson on a Commodore
Amiga.  In the process, several of the routines were rewritten to improve
efficiency and some new features added.  The program structure was changed
somewhat to make it easier to incorporate new devices. 

PLPLOT 4.0 is primarily the result of efforts by Maurice LeBrun and Geoff
Furnish of University of Texas at Austin to extend and improve the previous
work (PLPLOT 2.6b and 3.0, by Tony Richardson).  PLPLOT 4.0 is currently
used as the main graphics engine for TPC (Toroidal Particle Code), a large
plasma simulation code developed at the IFS \cite{lebrun89a}.  During this
work we have found that PLPLOT compares well with ``heavier'' packages
(read: expensive, slow) and is an excellent choice for scientists seeking
an inexpensive (or free) but high quality graphics package that runs on many
different computing platforms.

Some of the improvements in PLPLOT 4.0 include: the addition of several new
routines to enhance usage from Fortran and design of a portable C to
Fortran interface.  Additional support was added for coordinate mappings in
contour plots and some bugs fixed.  New labelling options were added.  The
font handling code was made more flexible and portable.  A portable PLPLOT
metafile driver and renderer was developed, allowing one to create a
generic graphics file and do the actual rendering later (even on a
different system).  The ability to create family output files was added.
The internal code structure was dramatically reworked, with elimination of
global variables (for a more robust package), the drivers rewritten to
improve consistency, and the ability to maintain multiple output streams
added.  An XFig driver was added.  Other contributions include Clair
Nielsen's (LANL) X-window driver (very nice for high-speed color graphics)
and tektronix file viewer.  At present, Maurice LeBrun and Geoff Furnish
are the active developers and maintainers of PLPLOT. 

We have attempted to keep PLPLOT 4.0 backward compatible with previous
versions of PLPLOT.  However, some functions are now obsolete, and many new
ones have been added (e.g.  new contouring functions, variable get/set
routines, functions that affect label appearance).  Codes written in C that
use PLPLOT must be recompiled including the new header file ({\tt
plplot.h}) before linking to the new PLPLOT library. 

PLPLOT is currently known to work on the following systems: SUNOS, HP-UX,
A/IX, DG/UX, UNICOS, Ultrix, VMS, Amiga/Exec, MS-DOS, OS/2, and NeXT, with
more expected.  The PLPLOT package is freely distributable, but {\em not\/}
in the public domain.  There have been various copyrights placed on the
software; see section \ref{sec:credits} for the full list and criteria for
distribution.  

We welcome suggestions on how to improve this code, especially in the form
of user-contributed enhancements or bug fixes.  If PLPLOT is used in any
published papers, please include an acknowledgment or citation of our work,
which will help us to continue improving PLPLOT.  Please direct all
communication to:

\begin{tabbing}
01234567\=
	89012345678901234567890123456789\= \kill
%
	\>Dr. Maurice LeBrun		\>Internet:\\
	\>Institute for Fusion Studies	\>mjl@fusion.ph.utexas.edu\\
	\>University of Texas\\
	\>Austin, TX  78712\\
\\
	\>Geoff Furnish			\>Internet:\\
	\>Institute for Fusion Studies	\>furnish@fusion.ph.utexas.edu\\
	\>University of Texas\\
	\>Austin, TX  78712\\
\\
	\>Tony Richardson		\>Internet:\\
	\>184 Electrical Engineering	\>amr@egr.duke.edu\\
	\>Duke University\\
	\>Durham, NC 27706\\
\end{tabbing}

The original version of this manual was written by Sze Tan.

\section {Getting a copy of the PLPLOT package}

At present, the only mechanism we are providing for distribution of the
PLPLOT is by electronic transmission over the Internet.  We encourage
others to make it available to users without Internet access.  PLPLOT may
be obtained by {\tt ftp} from {\tt hagar.ph.utexas.edu} (128.83.179.27).
Login as user {\tt anonymous}, set file transfer type to binary, and get
the newest plplot archive in the {\tt pub/} subdirectory.  We will provide
PLPLOT in both {\tt zoo} and {\tt tar} archives; get whichever one you
prefer. 

\section {Installing and Using the PLPLOT Library}

The installation procedure is by necessity system specific; installation
notes for each system are provided in Appendix \ref{ap:sys}.  The procedure
requires that all of the routines be compiled and they are then usually
placed in a linkable library. 

After the library has been created, you can write your main program to make
the desired PLPLOT calls.  Example programs in both C and Fortran are
included as a guide (if calling from C, you must include {\tt plplot.h} into
your program; see Appendix \ref{ap:lang} for more details).  Plots generated
from the example programs are shown at the end of this work.

You will then need to compile your program and link it with the PLPLOT
library(s).  Again, please refer to the documentation specific to your
system for this.  Note that there may be more than one library available to
link with, such as single or double precision, with or without X window
libraries, IEEE floating point or Motorola FFP, etc.  Make sure you link to
the correct one for your program. 

\section {Organization of this Manual}

The PLPLOT library has been designed so that it is easy to write programs
producing graphical output without having to set up large numbers of
parameters.  However, more precise control of the results may be necessary,
and these are accomodated by providing lower-level routines which change
the system defaults.  In Chapter \ref{simple}, the overall process of
producing a graph using the high-level routines is described.  Chapter
\ref{advanced} discusses the underlying concepts of the plotting process
and introduces some of the more complex routines.  Chapter \ref{reference}
is the reference section of the manual, containing an alphabetical list of
the user-accessible PLPLOT functions with detailed descriptions. 

Because the PLPLOT kernel is written in C, standard C syntax is used in the
description of each PLPLOT function.  The C and Fortran language interfaces
are discussed in Appendix \ref{ap:lang}; look there if you have difficulty
interpreting the call syntax as described in this manual.  The meaning of
function (subroutine) arguments is typically the same regardless of whether
you are calling from C or Fortran (but there are some exceptions to this).
The arguments for each function are usually specified in terms of PLFLT and
PLINT --- these are the internal PLPLOT representations for integer and
floating point, and are typically a long and a float (or an INTEGER and a
REAL, for Fortran programmers).  See Appendix \ref{ap:lang} for more detail.

Also, you can use PLPLOT from C++ just as you would from C.  No special
classes are available at this time, just use it as any other procedural
type library.  Simply include {\tt plplot.h}, and invoke as you would from
C. 

The output devices supported by PLPLOT are listed in Appendix \ref{ap:dev},
along with description of the device driver--PLPLOT interface, metafile
output, family files, and vt100/tek4010 emulators.  In Appendix
\ref{ap:sys} the usage and installation for each system supported by PLPLOT
is described (not guaranteed to be entirely up-to-date; check the release
notes to be sure). 

\section {Credits}
\label{sec:credits}

PLPLOT 4.0 was created through the effort of many individuals and funding
agencies.  We would like to acknowledge the support (financial and
otherwise) of the following institutions:
\begin{description}
\item	The Institute for Fusion Studies, University of Texas at Austin
\item	The Scientific and Technology Agency of Japan
\item	Japan Atomic Energy Research Institute
\item	Duke University
\item	Universite de Nice
\item	National Energy Research Supercomputer Center
\item	Los Alamos National Labs
\end{description}

The authors disclaim all warranties with regard to this software, including
all implied warranties of merchantability and fitness, In no event shall
the authors be liable for any special, indirect or consequential damages or
any damages whatsoever resulting from loss of use, data or profits, whether
in an action of contract, negligence or other tortious action, arising out
of or in connection with the use or performance of this software. 

The PLPLOT source code, except header files and those files explicitly
granting permission, may not be used in a commercial software package
without consent of the authors.  You are allowed and encouraged to include
the PLPLOT object library and header files in a commercial package provided
that: (1) it is explicitly and prominently stated that the PLPLOT library
is freely available, and (2) the full copyrights on the PLPLOT package be
displayed somewhere in the documentation for the package. 

PLPLOT was first derived from the excellent PGPLOT graphics package by T.
J. Pearson.  All parts of PLPLOT not explicitly marked by a copyright are
assumed to derive sufficiently from the original to be covered by the
PGPLOT copyright:
\begin{verbatim}
***********************************************************************
*                                                                     *
*  Copyright (c) 1983-1991 by                                         *
*  California Institute of Technology.                                *
*  All rights reserved.                                               *
*                                                                     *
*  For further information, contact:                                  *
*     Dr. T. J. Pearson                                               *
*     105-24 California Institute of Technology,                      *
*     Pasadena, California 91125, USA                                 *
*                                                                     *
***********************************************************************
\end{verbatim}

The code in PLPLOT not derived from PGPLOT is Copyright 1992 by Maurice J.
LeBrun and Geoff Furnish of the University of Texas at Austin and Tony
Richardson of Duke University.  Unless otherwise specified, code written by
us as a part of this package may be freely copied, modified and
redistributed without fee provided that all copyright notices are preserved
intact on all copies and modified copies.

The startup code for plrender.c is from {\tt xterm.c} of the X-windows
Release 5.0 distribution, and we reproduce its copyright here:
\begin{verbatim}
Copyright 1987, 1988 by Digital Equipment Corporation, Maynard, Massachusetts,
and the Massachusetts Institute of Technology, Cambridge, Massachusetts.

                        All Rights Reserved

Permission to use, copy, modify, and distribute this software and its 
documentation for any purpose and without fee is hereby granted, 
provided that the above copyright notice appear in all copies and that
both that copyright notice and this permission notice appear in 
supporting documentation, and that the names of Digital or MIT not be
used in advertising or publicity pertaining to distribution of the
software without specific, written prior permission.  

DIGITAL DISCLAIMS ALL WARRANTIES WITH REGARD TO THIS SOFTWARE, INCLUDING
ALL IMPLIED WARRANTIES OF MERCHANTABILITY AND FITNESS, IN NO EVENT SHALL
DIGITAL BE LIABLE FOR ANY SPECIAL, INDIRECT OR CONSEQUENTIAL DAMAGES OR
ANY DAMAGES WHATSOEVER RESULTING FROM LOSS OF USE, DATA OR PROFITS,
WHETHER IN AN ACTION OF CONTRACT, NEGLIGENCE OR OTHER TORTIOUS ACTION,
ARISING OUT OF OR IN CONNECTION WITH THE USE OR PERFORMANCE OF THIS
SOFTWARE.
\end{verbatim}

Thanks are also due to the many contributors to PLPLOT, including:
\begin{description}
\item Tony Richardson: Creator of PLPLOT 2.6b, 3.0
\item Sam Paolucci (postscript driver)
\item Clair Nielsen (X driver and tektronix file viewer)
\item Tom Rokicki (IFF driver and Amiga printer driver)
\end{description}

Finally, thanks to all those who submitted bug reports and other
suggestions.
